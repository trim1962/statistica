\chapter{Medie}
\label{cha:Medie}
\begin{table}
	\centering
	\begin{tabular}{S[table-format=1.0]S[table-format=2.0]S[table-format=3.0]}
		\toprule
		{$x_{i}$}&{$n_{i} $}  &{$x_{i}\cdot n_{i}$}  \\ 
		\midrule
		6	& 8 & 48 \\
		3	&  6&  18\\ 
		4	&  5& 20 \\ 
		5	&  3& 15 \\
		\midrule 
		{Totali}	&22&  101  \\
		\bottomrule 
	\end{tabular} 
	\caption{Media aritmetica ponderata}
	\label{tab:MediaAritmteicaPonderata}
\end{table}
\section{Medie}
Il concetto di media aritmetica dovrebbe essere noto, nel seguito daremo alcune definizioni che dipenderanno anche da quanto è stato detto nel capitolo precedente.
\begin{defn}[Media aritmetica]
	Se abbiamo un insieme di dati $x_{1},x_{2},x_{3},\cdots,x_{n}$ la media aritmetica\index{Media!aritmetica} $M$ di questi valori  è  la somma dei valori diviso il numero di elementi.\[M=\dfrac{x_{1}+x_{2}+x_{3}+\cdots+x_{n}}{n}=\dfrac{\sum_{i=1}^{n}x_{i}}{n} \]
\end{defn}

Nel caso di mentre un insieme  di frequenze assolute avremo la definizione:
\begin{defn}[Media aritmetica ponderata]
Se abbiamo una distribuzione di frequenze di n dati e un valore $x_{i}$ compare $n_{i}$ volte,  allora la media aritmetica ponderata\index{Media!aritmetica!ponderata}  $M$ dei valori è: \[M=\dfrac{x_{1}\cdot n_{1}+x_{2}\cdot n_{2}+\cdots+x_{n}\cdot n_{n}}{n_{1}+n_{2}+\cdots+n_{n} }=\dfrac{\sum_{i=1}^{n}x_{i}\cdot n_{i}}{\sum_{i=1}^{n} n_{i}}=\dfrac{\sum_{i=1}^{n}x_{i}\cdot n_{i}}{n}\]
\end{defn}
Il calcolo della media aritmetica non porta a grossi problemi, supponiamo di avere otto misure di velocità \[x_{i}=\SIlist[list-separator = {;}]{4.2;3.4;2.3;5.4;3.4;3.5;4.3;3.3}{\metre\per\second}\] La media $M$ è \[M=\dfrac{\num{4.2}+\num{3.4}+\num{2.3}+\num{5.4}+\num{3.4}+\num{3.5}+\num{4.3}+\num{3.3}} {8}=\SI{3.725}{\metre\per\second}\]

Per calcolare la media ponderata possiamo seguire il seguente esempio. La tabella riporta dei dati di un esperimento assieme alle frequenze assolute con cui essi compaiono. 
\begin{center}
\begin{tabular}{l*{4} {S[table-format=1.0]}}
	{$x_{i}=$}	&6  &3  &4  &5  \\
	\midrule 
	{$n_{i}=$}	& 8 &6  & 5 & 3 \\ 
\end{tabular}
\end{center}
Costruiamo ora  la~\vref{tab:MediaAritmteicaPonderata}.

Sommando l'ultima colonna della tabella otteniamo la media aritmetica ponderata segue 
\[M=\dfrac{\sum_{i=1}^{q}x_{i}\cdot n_{i}}{n}=\dfrac{\num{101}}{\num{22}} \approx\num{4.590} \]
La~\vref{tab:MediaAritmteicaPonderataExcel} permette di riprodurre l'esempio.

