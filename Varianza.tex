% !TeX root = statistica.tex
\chapter{Varianza}
\begin{thm}[Varianza]\index{Varianza}
	Se abbiamo una distribuzione di frequenze di n dati e un dato $x_{i}$ compare $n_{i}$ volte, la varianza è: \[\sigma^{2}=\dfrac{\sum_{i=1}^{n}x_{i}^{2}\cdot n_{i}}{\sum_{i=1}^{n} n_{i}}-M^2\] 
\end{thm}
\begin{proof}
\begin{align*}
\sigma^{2}=&\dfrac{\sum_{i=1}^{n}(x_{i}-M)^{2}\cdot n_{i}}{\sum_{i=1}^{n} n_{i}}\\
=&\dfrac{\sum_{i=1}^{n}(x_{i}^{2} -2Mx_{i}+M^{2})\cdot n_{i}}{\sum_{i=1}^{n} n_{i}}\\
=&\dfrac{\sum_{i=1}^{n}x_{i}^{2}\cdot n_{i}}{\sum_{i=1}^{n} n_{i}}-2 \dfrac{\sum_{i=1}^{n}Mx_{i}\cdot n_{i}}{\sum_{i=1}^{n} n_{i}} +\dfrac{\sum_{i=1}^{n}M^{2}\cdot n_{i}}{\sum_{i=1}^{n} n_{i}}\\
=&\dfrac{\sum_{i=1}^{n}x_{i}^{2}\cdot n_{i}}{\sum_{i=1}^{n} n_{i}}-2M \dfrac{\sum_{i=1}^{n}x_{i}\cdot n_{i}}{\sum_{i=1}^{n} n_{i}} +M^{2}\dfrac{\sum_{i=1}^{n}\cdot n_{i}}{\sum_{i=1}^{n} n_{i}}\\
=&\dfrac{\sum_{i=1}^{n}x_{i}^{2}\cdot n_{i}}{\sum_{i=1}^{n} n_{i}}-2MM +M^{2}\\
=&\dfrac{\sum_{i=1}^{n}x_{i}^{2}\cdot n_{i}}{\sum_{i=1}^{n} n_{i}}-2M^{2} +M^{2}\\
=&\dfrac{\sum_{i=1}^{n}x_{i}^{2}\cdot n_{i}}{\sum_{i=1}^{n} n_{i}}-M^{2}\\
\end{align*}
Come si voleva dimostrare
\end{proof}
Quindi la varianza è uguale alla media dei quadrati meno il quadrato della media