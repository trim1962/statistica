% !TeX root = statistica.tex
\chapter{Dimostrazioni}
\begin{thm}[Varianza]\index{Varianza}
	Se abbiamo una distribuzione di frequenze di n dati  $x_{i}$ di frequenza $n_{i}$, la varianza è: \[\sigma^{2}=\dfrac{\sum_{i=1}^{n}x_{i}^{2}\cdot n_{i}}{\sum_{i=1}^{n} n_{i}}-M^2\] 
\end{thm}
\begin{proof}
\begin{align*}
\sigma^{2}=&\dfrac{\sum_{i=1}^{n}(x_{i}-M)^{2}\cdot n_{i}}{\sum_{i=1}^{n} n_{i}}\\
=&\dfrac{\sum_{i=1}^{n}(x_{i}^{2} -2Mx_{i}+M^{2})\cdot n_{i}}{\sum_{i=1}^{n} n_{i}}\\
=&\dfrac{\sum_{i=1}^{n}x_{i}^{2}\cdot n_{i}}{\sum_{i=1}^{n} n_{i}}-2 \dfrac{\sum_{i=1}^{n}Mx_{i}\cdot n_{i}}{\sum_{i=1}^{n} n_{i}} +\dfrac{\sum_{i=1}^{n}M^{2}\cdot n_{i}}{\sum_{i=1}^{n} n_{i}}\\
=&\dfrac{\sum_{i=1}^{n}x_{i}^{2}\cdot n_{i}}{\sum_{i=1}^{n} n_{i}}-2M \dfrac{\sum_{i=1}^{n}x_{i}\cdot n_{i}}{\sum_{i=1}^{n} n_{i}} +M^{2}\dfrac{\sum_{i=1}^{n}\cdot n_{i}}{\sum_{i=1}^{n} n_{i}}\\
=&\dfrac{\sum_{i=1}^{n}x_{i}^{2}\cdot n_{i}}{\sum_{i=1}^{n} n_{i}}-2MM +M^{2}\\
=&\dfrac{\sum_{i=1}^{n}x_{i}^{2}\cdot n_{i}}{\sum_{i=1}^{n} n_{i}}-2M^{2} +M^{2}\\
=&\dfrac{\sum_{i=1}^{n}x_{i}^{2}\cdot n_{i}}{\sum_{i=1}^{n} n_{i}}-M^{2}\\
\end{align*}
Come si voleva dimostrare
\end{proof}
Quindi la varianza è uguale alla media dei quadrati meno il quadrato della media.
Consideriamo un esempio 
Poniamo di avere la seguente distribuzione di dati
\begin{center}
	\begin{tabular}{l*{4} {S[table-format=1.0]}}
		{$x_{i}=$}	&3  &7  &9  &10   \\
		\midrule 
		{$n_{i}=$}	&3 &4  &5 & 2 \\   
	\end{tabular}
\end{center}
Calcoliamo la varianza con il vecchio metodo e otteniamo la~\vref{tab:CalcoloVarianzaPonderataOld} da cui segue che la media è: \[M=\num{7.5}\] mentre la varianza vale:
\[\sigma^2=\num{6.96}\]
Con il secondo costruiamo la~\cref{tab:CalcoloVarianzaPonderatanew}. La Tabella ci permette di calcolare la media dei valori e la media dei quadrati. Possiamo scrivere: la media è: \[M=\num{7.5}\] il cui quadrato è \[M^{2}=56.21\] mentre la media dei quadrati è:
\[M_{q}=\dfrac{885}{14}\backsimeq 63.21\]
\begin{table}
	\centering
	\begin{tabular}{S[table-format=1.0]S[table-format=2.0]S[table-format=2.0]S[table-format=1.2]S[table-format=1.4]S[table-format=2.4] }
		\toprule
		{$x_i$}	&{$n_i$}  & {$x_i\cdot n_i$} & {$x_i-M$} & {$(x_i-M )^2$} & {$(x_i-M)^2\cdot n_i $ } \\
		\midrule 
		3	& 3 & 9 & -4.5 &20.25  & 60.75 \\ 
		7	& 4 & 28 & -0.5 &0.25  & 1 \\ 
		9	& 5 & 50 & 2.5 &6.25  & 31.25 \\ 
		10	& 2 & 18 & 1.5&2.25  & 4.5 \\ 
		\midrule 
		{Totali}& 14& 105 &  &  & 97.5 \\
		\bottomrule 
	\end{tabular} 
	\caption{Calcolo varianza ponderata}
	\label{tab:CalcoloVarianzaPonderataOld}
\end{table}
\begin{table}
	\centering
	\begin{tabular}{S[table-format=1.0]S[table-format=2.0]S[table-format=2.0]S[table-format=2.0]S[table-format=2.0]}
		\toprule
		{$x_i$}	&{$n_i$}  & {$x_i\cdot n_i$} & {$x_{i}^2$} & {$x_{i}^2\cdot n_{i}$} \\
		\midrule 
		3	& 3 & 9 & 9&27   \\ 
		7	& 4 & 28 & 49 &196  \\ 
		9	& 5 & 50 & 100 &500   \\ 
		10	& 2 & 18 & 81&162\\ 
		\midrule 
		{Totali}& 14& 105 &  & 885 \\
		\bottomrule 
	\end{tabular} 
	\caption{Calcolo varianza ponderata}
	\label{tab:CalcoloVarianzaPonderatanew}
\end{table}
\begin{thm}[Scarto dalla media]\index{Media!scarto}
Lo scarto dalla media vale zero.\[\sum_{i=1}^{n}(x_{i}-M)=0\] 
\end{thm}
\begin{proof}
	\begin{align*}
	\sum_{i=1}^{n}(x_{i}-M)=&\\
	=&\sum_{i=1}^{n}x_{i}-\sum_{i=1}^{n}M\\
	=&nM-nM=0\\
	\end{align*}
	Come volevasi dimostrare
\end{proof}