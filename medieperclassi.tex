\chapter{Medie per classi}
I dati possono essere raggruppati per intervalli in questo caso si dice che la distribuzione è suddivisa per classi di frequenza\index{Frequenza!classi}, per calcolare media e varianza occorre  applicare qualche accorgimento. Consideriamo la~\vref{tab:Classifrequenza} per calcolare la media introduciamo il concetto di valore centrale. 
\begin{defn}[Valore centrale]
	Se abbiamo un intervallo di frequenza $[a,b]$ definiamo valore centrale dell'intervallo \[m_i=\dfrac{a+b}{2}\]
\end{defn}\index{Valore!centrale}
Con questa definizione otteniamo la~\vref{tab:Valorecentrale}
\begin{table}
	\centering
	\begin{tabular}{cS[table-format=1.0] }
		\toprule
		{$Intervallo$}	  & {$n_i$}  \\
		\midrule 
		10-20	 & 5   \\ 
		20-30	 & 10    \\ 
		30-40	 & 12    \\ 
		\bottomrule 
	\end{tabular} 
	\caption{Classi di frequenza}
	\label{tab:Classifrequenza}
\end{table}
\begin{table}
	\centering
	\begin{tabular}{cS[table-format=1.0]S[table-format=1.0]S[table-format=1.0]S[table-format=2.0]S[table-format=2.0]S[table-format=2.0]S[table-format=2.0]}
		\toprule
		{$Intervallo$}	  & {$m_i$}&{$n_i$}&{$m_i\cdot n_i$}&{$m_i-M$}& {$(m_i-M)^2$} &{$(m_i-M)^2n_i$}\\
		\midrule 
		10-20&15	 & 5 &75& 12.59&158.5081&792.5405 \\ 
		20-30&25	 & 10&250 &-2.59&6.7081&67.081   \\ 
		30-40&35	 & 12&420 &7.41&54.9081&658.8972   \\
		\midrule 
		Totale&	 &27 &745 & & &1518.5187  \\
		\bottomrule 
	\end{tabular} 
	\caption{Varianza per classi}
	\label{tab:Varianzaperclassi}
\end{table}
\begin{table}
	\centering
	\begin{tabular}{cS[table-format=1.0]S[table-format=1.0]}
		\toprule
		{$Intervallo$}	  & {$m_i$}&{$n_i$}  \\
		\midrule 
		10-20&15	 & 5   \\ 
		20-30&25	 & 10    \\ 
		30-40&35	 & 12    \\ 
		\bottomrule 
	\end{tabular} 
	\caption{Valore centrale}
	\label{tab:Valorecentrale}
\end{table}
\begin{defn}[Media aritmetica ponderata per classi]
	Se abbiamo una distribuzione di $n$ classi di frequenze  e se $m_i$ è il valore centrale di ciascuna classe con frequenza $n_i$,  allora la media aritmetica ponderata\index{Media!aritmetica!ponderata}  $M$ dei valori è: \[M=\dfrac{\sum_{i=1}^{n}m_{i}\cdot n_{i}}{\sum_{i=1}^{n} n_{i}}=\dfrac{\sum_{i=1}^{n}m_{i}\cdot n_{i}}{n}\]
\end{defn}
Possiamo quindi costruire la~\vref{tab:Mediaperclassi} e ottenere la media\[M=\dfrac{745}{27}\simeq\num{27.59}\]
\begin{defn}[Varianza ponderata per classi]
		Se abbiamo una distribuzione di $n$ classi di frequenze  e se $m_i$ è il valore centrale di ciascuna classe con frequenza $n_i$
	la varianza  è la media della somma dei quadrato degli scarti dalla media\[\sigma^{2}=\dfrac{\sum_{i=1}^{n}(m_{i}-M)^{2}\cdot n_{i}}{\sum_{i=1}^{n} n_{i}}\] 
\end{defn}
Otteniamo la~\vref{tab:Varianzaperclassi} per cui \[\sigma^2=\dfrac{1518.5187}{27}\simeq\num{56.24}\]
mentre lo scarto quadratico 
\[\sigma=\sqrt{56.24}\simeq\num{7.50}\]
\begin{table}
	\centering
	\begin{tabular}{cS[table-format=1.0]S[table-format=1.0]S[table-format=1.0]}
		\toprule
		{$Intervallo$}	  & {$m_i$}&{$n_i$}&{$m_i\cdot n_i$}  \\
		\midrule 
		10-20&15	 & 5 &75  \\ 
		20-30&25	 & 10&250    \\ 
		30-40&35	 & 12&420    \\
		\midrule 
		Totale&	 &27 &745    \\
		\bottomrule 
	\end{tabular} 
	\caption{Media per classi}
	\label{tab:Mediaperclassi}
\end{table} 