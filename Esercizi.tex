\chapter{Esercizi}
\sisetup{%
	range-phrase = {\ \linebreak[0]\text{to}\ \nolinebreak},
	list-separator = {\text{, }},
	list-final-separator = {,\ \linebreak[0]\text{e }},
	list-pair-separator = {\ \text{and}\ },
	list-separator = {,\ \linebreak[0]}
}%
\begin{enumerate}
	\item Durante una misurazione vengono riportate le seguenti altezze \SIlist{164; 151; 170;  150; 169; 172; 166; 178; 156; 172; 174; 161; 168; 160; 156; 179,
	180; 172; 152; 172; 175; 166; 174; 150; 169; 150; 169}{\cm}. Determinarne  le frequenze relative.
\item Due classi effettuano la medesima prova. La prima ottiene i seguenti voti: \numlist{3; 3; 4; 2; 2; 7; 6; 1; 6; 2; 3; 8; 8; 6; 1; 7; 7; 7; 8; 1; 9; 8; 3; 7; 7; 8; 5}, la seconda \numlist{9; 7; 4; 3; 6; 5; 3; 4; 2; 7; 6; 6; 1; 4; 5; 2; 2; 2; 1; 8; 8; 2; 7; 6; 3; 10; 7; 6; 9;
	5; 6; 6; 4}. Confrontare le frequenze percentuali delle due prove.
\end{enumerate}