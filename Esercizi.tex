\chapter{Esercizi}
\sisetup{%
	range-phrase = {\ \linebreak[0]\text{to}\ \nolinebreak},
	list-separator = {\text{, }},
	list-final-separator = {,\ \linebreak[0]\text{e }},
	list-pair-separator = {\ \text{and}\ },
	list-separator = {,\ \linebreak[0]}
}%
\begin{enumerate}
	\item Durante una misurazione vengono riportate le seguenti altezze \SIlist{164; 151; 170;  150; 169; 172; 166; 178; 156; 172; 174; 161; 168; 160; 156; 179,
	180; 172; 152; 172; 175; 166; 174; 150; 169; 150; 169}{\cm}. Determinarne  le frequenze relative.
\item Due classi effettuano la medesima prova. La prima ottiene i seguenti voti: \numlist{3; 3; 4; 2; 2; 7; 6; 1; 6; 2; 3; 8; 8; 6; 1; 7; 7; 7; 8; 1; 9; 8; 3; 7; 7; 8; 5}, la seconda \numlist{9; 7; 4; 3; 6; 5; 3; 4; 2; 7; 6; 6; 1; 4; 5; 2; 2; 2; 1; 8; 8; 2; 7; 6; 3; 10; 7; 6; 9;
	5; 6; 6; 4}. Confrontare le frequenze percentuali delle due prove.
\item Date le seguenti altezze \SIlist{1.88; 1.88; 1.61; 1.89; 1.74; 1.64; 2; 1.84; 1.95; 1.73; 1.80; 1.96; 1.77; 1.90; 1.76; 1.71; 1.90;
	1.92; 1.91; 1.89; 1.64; 1.81; 2; 1.64; 1.74; 1.67; 1.89; 1.95; 1.66; 1.96; 1.82; 1.94}{\m} calcolarne media e scarto quadratico medio.
\item Dati i seguenti lanci di dadi, \numlist{1; 1; 5; 6; 3; 1; 2; 5; 2; 3; 2; 2; 1; 6; 1; 3; 4; 4; 1; 6; 1; 6; 3; 6; 6; 4; 4; 1; 5; 5}, calcolarne moda, mediana media.
\item Viene fatto un sondaggio, viene chiesto di esprime un giudizio che varia da zero pessimo a quattro  ottimo. La tabella esprime il sondaggio
\begin{center}
	\begin{tabular}{l*{5} {S[table-format=3.0]}}
		{Giudizio}	&0  &1  &2  &3&4  \\
		\midrule 
		{Frequenze}	& 100 &120  & 130 & 110&105 \\ 
	\end{tabular}
\end{center} Calcolare media, mediana, moda, varianza, deviazione standard.
\end{enumerate}