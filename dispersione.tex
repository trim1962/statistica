\chapter{Indici di dispersione}
\section{Intervalli}
La più semplice misura della dispersione di un campione è l'intervallo di variazione.
\begin{defn}[Intervallo di variazione]
	Dato un insieme di dati  $\lbrace x_{1},x_{2},\cdots,x_{n}\rbrace$, l'intervallo di variazione\index{Intervallo!variazione} è la differenza tra il valore massimo e il valore minimo della distribuzione 
	\begin{align*}
	x_{Max}=&\max\lbrace x_{1},x_{2},\cdots,x_{n}\rbrace\\
	x_{Min}=&\min\lbrace x_{1},x_{2},\cdots,x_{n}\rbrace\\
	I_v=&x_{Max}-x_{Min}\\
	\end{align*}
\end{defn}
In pratica se abbiamo questa distribuzione
\begin{center}
	\begin{tabular}{l*{2} {S[table-format=1.0]}S[table-format=2.0]*{2} {S[table-format=1.0]}}
		{$x_{i}=$}	&6  &7  &20  &8  &4 \\  
	\end{tabular}
\end{center}
avremo che \[I_v=20-4=16\]
\section{Scarti}
\subsection{Scarti semplici}
\begin{defn}[Scarti semplici]
Dato un insieme di dati  $\lbrace x_{1},x_{2},\cdots,x_{n}\rbrace$, diremo scarti semplici\index{Scarto!semplice} da un valore prefissato $D$,  i valori \[d_{i}=x_{i}-D \]
\end{defn}
Il calcolo degli scarti non è complicato. L'esempio seguente mostra come procedere. Se abbiamo cinque dati, $x_{i}=\numlist{4;5.4;3;2;3.4}$, per calcolare gli scarti da $D=3$, impostiamo  la~\vref{tab:CalcoloScartiSemplici}. La~\vref{tab:ScartiSempliciExcel} permette di calcolare lo scarto. 
\begin{table}
	\centering
	\begin{tabular}{S[table-format=1.1]S[table-format=1.0]S[table-format=1.1]}
		\toprule
		{$x_{i}$}& {$D$} &{$d_{i}=x_{i}-D$}  \\ 
		\midrule
		4&  3& 1 \\ 
		5.4& 3 &2.4  \\ 
		3&  3& 0 \\ 
		2&  3& -1 \\ 
		3.4&3 &0.4\\
		\bottomrule
	\end{tabular} 
	\caption{Scarti semplici}
	\label{tab:CalcoloScartiSemplici}
\end{table}
\begin{defn}[Scarti ponderati]
Per una distribuzione di frequenze di n dati $\lbrace x_{1},x_{2},\cdots,x_{n}\rbrace$ $x_{i}$ con frequenza  $n_{i}$,  diremo scarto ponderato\index{Scarto!semplice!ponderato} da un valore prefissato $D$,  i valori \[d_{i}=(x_{i}-D)\cdot n_{i} \]
\end{defn}
Supponiamo di avere questi dati 
\begin{center}
	\begin{tabular}{l*{4} {S[table-format=1.0]}}
		{$x_{i}=$}	&5  &3  &7  &4  \\
		\midrule 
		{$n_{i}=$}	& 7 &2  & 4 & 8 \\ 
	\end{tabular}
\end{center}
Per calcolare gli  scarti dal valore $D=8$  procederemo come mostrato nel foglio elettronico mostrato nella~\vref*{tab:ScartiPonderatiExcel} che permette di costruire la~\vref{tab:ScartiPonderati}
\begin{table}
	\centering
	\begin{tabular}{*{4} {S[table-format=1.0]}S[table-format=2.0]}
		\toprule
		{$x_{i}$} & {$n_{i}$ } & {$D$} & {$x_{i}-D $} & {$d_{i}=(x_{i}-D ) \cdot n_{i}$} \\ 
		\midrule
		5& 7 &8  & -3 & -21 \\ 
		3&  2& 8 &  -5& -15 \\ 
		7&  4& 8 &  -1&  -4\\ 
		4&  8& 8 & -4 & -32 \\ 
		\bottomrule
	\end{tabular} 
	\caption{Scarti ponderati}
	\label{tab:ScartiPonderati}
\end{table}

\begin{defn}[Scarto assoluto]
Dato un insieme di dati  $\lbrace x_{1},x_{2},\cdots,x_{n}\rbrace$, diremo scarto assoluto\index{Scarto!assoluto} da un valore prefissato $D$,  i valori \[d_{i}= \abs{x_{i}-D }\]
\end{defn}
\subsection{Scarti medi}
Per indicare la variabilità dei valori,  possiamo utilizzare vari indicatori. Un primo indicatore è lo scarto medio assoluto\index{Scarto!medio!assoluto} dalla media
\begin{defn}[Scarto medio assoluto]
Dato un insieme di dati  $\lbrace x_{1},x_{2},\cdots,x_{n}\rbrace$ diremo scarto medio assoluto dalla media $M$  il valore \[ S_{M}=\dfrac{\sum_{i=1}^{n}\abs{x_{i}-M}}{n}\]
\end{defn}
Supponiamo di avere i seguenti dati $x_{i}=\numlist{3;7;8;10;2}$. Per calcolare lo $S_{M}$, costruiamo la~\vref{tab:ScartoMedioAssoluto}. La somma dei dati della prima colonna diviso il numero di elementi fornisce la media $M=\dfrac{30}{5}=6$.  

Il valore ottenuto permette di costruire la seconda colonna. Lo scarto medio assoluto  l'ottengo sommando i valori della colonna diviso il numero di elementi e otteniamo  $S_{M}=\dfrac{14}{5}=\num{2.8}$. Il foglio di calcolo presente nella~\vref{tab:ScartoMedioAssolutoExcel} esprime quanto detto.
\begin{table}
	\centering
	\begin{tabular}{l*{4} {S[table-format=2.0]}}
		\toprule
	&{$x_{i}$}&{$M$}&{$x_{i}-M$}	& {$\abs{x_{i}-M} $} \\
	\midrule 
		&3&6&-3&3\\ 
		&7&6&1&1  \\ 
		&8&6&2& 2 \\ 
		&10&6&4&4  \\ 
		&2&6&-4& 4 \\ 
		\midrule
		{Totali}&30&&&14  \\
		\bottomrule 
	\end{tabular} 
	\caption{Scarto medio assoluto}
	\label{tab:ScartoMedioAssoluto}
\end{table}
Per le distribuzioni vale questa definizione:
\begin{defn}[Scarto medio assoluto ponderato]
	Se abbiamo una distribuzione di frequenze di n dati  $x_{i}$ di frequenza $n_{i}$ volte,  lo scarto medio assoluto~\index{Scarto!medio!assoluto ponderato} è  il valore \[ S_{M}=\dfrac{\sum_{i=1}^{n}\abs{x_{i}-M}\cdot n_{i} }{  \sum_{i=1}^{n} n_{i}}\]
\end{defn}
Poniamo di avere la seguente distribuzione di dati:
\begin{center}
	\begin{tabular}{l*{2} {S[table-format=1.0]}S[table-format=2.0]*{2} {S[table-format=1.0]}}
		{$x_{i}=$}	&3  &7  &30  &5  &4 \\
		\midrule 
		{$n_{i}=$}	& 7 &2  & 4 & 8 & 9\\   
	\end{tabular}
\end{center}
Nell'esempio che segue costruiamo la~\vref{tab:ScartoMedioAssolutoPonderato}.  La somma della seconda colonna è il numero degli elementi, dividendo la somma della terza per il numero degli elementi da la media $M=\dfrac{\num{120}}{\num{20}}=\num{6}$. 

Conoscendo la media otteniamo la colonna quattro, ricordando che, essendo la differenza in valore assoluto, le quantità che ottengo sono sempre positive. Infine otteniamo la colonna cinque, la cui somma divisa il numero degli elementi è lo scarto medio assoluto dalla media $S_{M}=\dfrac{\num{58}}{\num{20}}=\num{2.9}$. 

La~\vref{tab:ScartiMedioAssolutoPonderatiExcel} spiega come costruire un piccolo foglio di calcolo per calcolare lo scarto.
\begin{table}
	\centering
	\begin{tabular}{*{2} {S[table-format=2.0]}S[table-format=3.0]*{2} {S[table-format=2.0]}}
	\toprule
	{$x_{i}$}&{$n_{i}$} &{$x_{i}\cdot n_{i}$}  &{$\abs{x_{i}-M}$}  &{$\abs{x_{i}-M}\cdot n_{i} $}  \\
	\midrule 
	3		& 3 &9  & 3 &  9\\ 
	7		& 5 & 35 & 1 &  5\\ 
	30		&  1& 30 &  24&  24\\ 
	5		& 2 & 10 &  1&  2 \\ 
	4		& 9 & 36 &  2& 18 \\
	\midrule 
	{Tot.}&  20& 120 &  & 58 \\
	\bottomrule 
	\end{tabular} 
	\caption{Scarto medio assoluto ponderato}
	\label{tab:ScartoMedioAssolutoPonderato}
\end{table}

\subsection{Varianza e scarto quadratico medio}
Un altro tipo di scarto è la varianza.\index{Varianza} Della varianza diamo due definizioni a seconda che i dati siano o non siano ponderati.
\begin{defn}[Varianza]
Se abbiamo un insieme di dati $\lbrace x_{1},x_{2},\cdots,x_{n}\rbrace$, la varianza  è la media della somma dei quadrati degli scarti dalla media\[\sigma^{2}=\dfrac{\sum_{i=1}^{n}(x_{i}-M)^2}{n}  \] 
\end{defn}
Un esempio di calcolo della varianza è il seguente. 
Se abbiamo i seguenti dati $x_{i}=\numlist{1;5;4;6;9}$ possiamo definire la~\vref{tab:CalcoloVarianza}. La somma degli elementi della prima colonna, diviso il numero dei dati che la compongono ci permette  di determinare la media $M=\dfrac{\num{25}}{\num{5}}=\num{5}$. 
La media ottenuta serve per  calcolare la seconda colonna. Per ottenere la varianza, sommiamo gli elementi della terza colonna diviso il numero di essi e otteniamo $\sigma^2=\dfrac{34}{5}=\num{6,8}$. La~\vref{tab:CalcoloVarianzaExcel} illustra l'esempio.
\begin{table}
	\centering
	\begin{tabular}{l*{3}{S[table-format=2.0]}}
		\toprule
		& {$x_i$} & {$x_i-M$} &{$(x_i-M )^2$}  \\ 
		\midrule
			&  1& -4 & 16 \\ 
			&  5&  0&  0\\ 
			&  4&  -1&  1\\ 
			&  6&  1&  1\\ 
			&  9&  1&  16\\ 
		\midrule
	{Totali}	& 25 &  &34  \\ 
		\bottomrule
	\end{tabular} 
	\caption{Calcolo varianza}
	\label{tab:CalcoloVarianza}
\end{table}
\begin{defn}[Varianza ponderata]\index{Varianza!ponderata}
		Se abbiamo una distribuzione di frequenze di n dati  $x_{i}$ di frequenza $n_{i}$ volte, la varianza  è la media della somma dei quadrato degli scarti dalla media\[\sigma^{2}=\dfrac{\sum_{i=1}^{n}(x_{i}-M)^{2}\cdot n_{i}}{\sum_{i=1}^{n} n_{i}}\] 
\end{defn}
Poniamo di avere la seguente distribuzione di dati
\begin{center}
	\begin{tabular}{l*{5} {S[table-format=1.0]}}
		{$x_{i}=$}	&2  &5  &3  &6  &8 \\
		\midrule 
		{$n_{i}=$}	& 4 &5  & 6 & 2 & 3\\   
	\end{tabular}
\end{center}
Per ottenere una varianza ponderata costruiamo la~\vref{tab:CalcoloVarianzaPonderata}. La somma degli elementi della seconda colonna ci dice quanti sono i dati. Per ottenere la media dividiamo la somma degli elementi della terza colonna con il numero dei dati e otteniamo: $M=\dfrac{87}{20}=\num{4.35}$. 

Conoscendo la media $M$  compiliamo le rimanenti colonne. La somma dell'ultima colonna divisa per il numero degli elementi ci permette di  calcolare la varianza $\sigma^2=\dfrac{\num{35.45}}{5}=\num{7.09}$.
\begin{table}
	\centering
	\begin{tabular}{S[table-format=1.0]S[table-format=2.0]S[table-format=2.0]S[table-format=1.2]S[table-format=1.4]S[table-format=2.4] }
		\toprule
	{$x_i$}	&{$n_i$}  & {$x_i\cdot n_i$} & {$x_i-M$} & {$(x_i-M )^2$} & {$(x_i-M)^2\cdot n_i $ } \\
		\midrule 
		2	& 4 & 8  & 0.35 &0.1225  & 0.49 \\ 
		5	& 5 & 25 & 0.65 &0.4225  & 2.1125 \\ 
		3	& 6 & 18 & 1.65 &2.7225  & 16.335 \\ 
		6	& 2 & 12 & -2.35&5.5225  & 11.045 \\ 
		8	& 3 & 24 & 1.35 &1.8225  & 5.4675 \\
		\midrule 
	{Totali}& 20& 87 &  &  & 35.45 \\
		\bottomrule 
	\end{tabular} 
	\caption{Calcolo varianza ponderata}
	\label{tab:CalcoloVarianzaPonderata}
\end{table}

La varianza,  dal punto di vista dimensionale, è un problema.  Essendo somma di quadrati, ha anche le dimensioni al quadrato. Per mantenere le dimensioni omogenee con i dati  di partenza, si usa lo scarto quadratico medio\index{Scarto!quadratico!medio}. 
\begin{table}
	\centering
	\begin{tabular}{S[table-format=1.0]S[table-format=2.0]S[table-format=2.0]S[table-format=1.2]S[table-format=1.4]S[table-format=2.4] }
		\toprule
		{$x_i$}	&{$n_i$}  & {$x_i\cdot n_i$} & {$x_i-M$} & {$(x_i-M )^2$} & {$(x_i-M)^2\cdot n_i $ } \\
		\midrule 
		4	& 8 & 32  & 1.19 &1.40  & 11.24 \\ 
		0	& 7 & 0 & -2.81 &7.92  & 55.46 \\ 
		2	& 6 & 12 & -0.81 &0.66  & 3.98 \\ 
		5	& 4 & 20& 2.19&4.78   & 19.10 \\ 
		6	& 2 & 12 & 3.19 &10.15  &20.29 \\
		\midrule 
		{Totali}& 27& 76 &  &  & 110.07 \\
		\bottomrule 
	\end{tabular} 
	\caption{Calcolo scarto quadratico medio}
	\label{tab:CalcoloScartoQuadratico}
\end{table}
\begin{defn}[Scarto quadratico medio]
	Se abbiamo un insieme di dati $\lbrace x_{1},x_{2},\cdots,x_{n}\rbrace$, lo scarto quadratico medio è  uguale alla radice quadrata della varianza \[\sigma=\sqrt{\sigma^2} \]
	Avremo che 
	\begin{align*}
		\sigma=\sqrt{\dfrac{\sum_{i=1}^{n}(x_{i}-M)^2}{n}}&&\sigma=\sqrt{\dfrac{\sum_{i=1}^{n}(x_{i}-M)^{2}\cdot n_{i}}{\sum_{i=1}^{n} n_{i}}}
	\end{align*}
\end{defn}
Concludiamo con un esempio. Poniamo di avere la seguente distribuzione di dati
\begin{center}
	\begin{tabular}{l*{5} {S[table-format=1.0]}}
		{$x_{i}=$}	&4  &0  &2  &5  &6\\
		\midrule 
		{$n_{i}=$}	&8 &7  &6 &4 &2\\   
	\end{tabular}
\end{center}
Costruiamo la~\vref{tab:CalcoloScartoQuadratico}. Otteniamo la media  $M=\dfrac{76}{27}\simeq\num{2.81}$.  valore ci permette di completare la tabella. Sommando l'ultima colonna e dividendola per il  numero numer degli elementi otteniamo la varianza $\sigma^2=\dfrac{110.07}{27}\simeq\num{4.23}$
\[\sigma=\sqrt{4.23}\simeq\num{2.06}\]
Per ottenere l'esempio si può utilizzare la~\vref{tab:CalcoloScartoQuadraticoMedioExcel}

\chapter{Un attimo di riflessione}
\begin{table}
	\centering
	\begin{tabular}{cS[table-format=1.0]S[table-format=1.0]S[table-format=1.0]}
		\toprule
		{$Intervallo$}	  & {$m_i$}&{$n_i$}&{$m_i\cdot n_i$}  \\
		\midrule 
		10-20&15	 & 5 &75  \\ 
		20-30&25	 & 10&250    \\ 
		30-40&35	 & 12&420    \\
		\midrule 
		Totale&	 &27 &745    \\
		\bottomrule 
	\end{tabular} 
	\caption{Media per classi}
	\label{tab:Mediaperclassi}
\end{table}
La poesia \emph{La statistica}
 di Trilussa ci permette di riflettere sul significato di quanto abbiamo definito e calcolato fino a questo momento.
	\begin{verse}
	Sai ched'è la statistica? È 'na cosa\\
	che serve pe' fa' un conto in generale\\
	de la gente che nasce, che sta male,\\
	che more, che va in carcere e che sposa.\\
	
	Ma pe' me la statistica curiosa\\
	è dove c'entra la percentuale,\\
	pe' via che, lì, la media è sempre eguale\\
	puro co' la persona bisognosa.\\
	
	Me spiego: da li conti che se fanno\\
	seconno le statistiche d'adesso\\
	risurta che te tocca un pollo all'anno:\\
	
	e, se nun entra ne le spese tue,\\
	t'entra ne la statistica lo stesso\\
	perché c'è un antro che ne magna due.\\
\end{verse}~\cite{2004tutte}

Dopo qualche spiegazioni fornisce qualche spunto di riflessione riguardo alla media. Abbiamo due persone la prima non mangia nessun pollo la seconda due. La media vale uno, lo scarto è uno come dalla~\vref{tab:PolliTrilussauno}.\par 
\begin{table}
	\centering
	\begin{tabular}{S[table-format=1.0]S[table-format=2.0]S[table-format=1.2] }
		\toprule
		{$x_i$}	  & {$x_i-M$} & {$(x_i-M )^2$}  \\
		\midrule 
		2	 & 1  & 1   \\ 
		0	 & -1 & 1   \\ 
		\midrule 
		{Totale}& & 2    \\
		\bottomrule 
	\end{tabular} 
	\caption{I polli di Trilussa uno}
	\label{tab:PolliTrilussauno}
\end{table}
Se le due persone mangiano un pollo ciascuno otteniamo la~\vref{tab:PolliTrilussadue}. Anche qui media vale uno ma lo scarto è zero.\par  Morale se Trilussa avesse considerato anche lo scarto il suo sonetto avrebbe un altro significato.
\begin{table}
	\centering
	\begin{tabular}{S[table-format=1.0]S[table-format=2.0]S[table-format=1.2] }
		\toprule
		{$x_i$}	  & {$x_i-M$} & {$(x_i-M )^2$}  \\
		\midrule 
		1	 & 0  & 0   \\ 
		1	 & 0 & 0   \\ 
		\midrule 
		{Totale}& & 0    \\
		\bottomrule 
	\end{tabular} 
	\caption{I polli di Trilussa due}
	\label{tab:PolliTrilussadue}
\end{table}
\section{Medie per classi}
I dati possono essere raggruppati per intervalli in questo caso si dice che la distribuzione è suddivisa per classi di frequenza\index{Frequenza!classi}. In questo caso per calcolare media e varianza occorre  applicare qualche accorgimento. Consideriamo la~\vref{tab:Classifrequenza} per calcolare la media introduciamo il concetto di valore centrale. 
\begin{defn}[Valore centrale]
	Se abbiamo un intervallo di frequenza $[a,b]$ definiamo valore centrale dell'intervallo \[m_i=\dfrac{a+b}{2}\]
\end{defn}\index{Valore!centrale}
Con questa definizione otteniamo la~\vref{tab:Valorecentrale}
\begin{table}
	\centering
	\begin{tabular}{cS[table-format=1.0] }
		\toprule
		{$Intervallo$}	  & {$n_i$}  \\
		\midrule 
		10-20	 & 5   \\ 
		20-30	 & 10    \\ 
		30-40	 & 12    \\ 
		\bottomrule 
	\end{tabular} 
	\caption{Classi di frequenza}
	\label{tab:Classifrequenza}
\end{table}
\begin{table}
	\centering
	\begin{tabular}{cS[table-format=1.0]S[table-format=1.0]S[table-format=1.0]S[table-format=2.0]S[table-format=2.0]S[table-format=2.0]S[table-format=2.0]}
		\toprule
		{$Intervallo$}	  & {$m_i$}&{$n_i$}&{$m_i\cdot n_i$}&{$m_i-M$}& {$(m_i-M)^2$} &{$(m_i-M)^2n_i$}\\
		\midrule 
		10-20&15	 & 5 &75& 12.59&158.5081&792.5405 \\ 
		20-30&25	 & 10&250 &-2.59&6.7081&67.081   \\ 
		30-40&35	 & 12&420 &7.41&54.9081&658.8972   \\
		\midrule 
		Totale&	 &27 &745 & & &1518.5187  \\
		\bottomrule 
	\end{tabular} 
	\caption{Varianza per classi}
	\label{tab:Varianzaperclassi}
\end{table}
\begin{table}
	\centering
	\begin{tabular}{cS[table-format=1.0]S[table-format=1.0]}
		\toprule
		{$Intervallo$}	  & {$m_i$}&{$n_i$}  \\
		\midrule 
		10-20&15	 & 5   \\ 
		20-30&25	 & 10    \\ 
		30-40&35	 & 12    \\ 
		\bottomrule 
	\end{tabular} 
	\caption{Valore centrale}
	\label{tab:Valorecentrale}
\end{table}
\begin{defn}[Media aritmetica ponderata per classi]
	Se abbiamo una distribuzione di $n$ classi di frequenze  e se $m_i$ è il valore centrale di ciascuna classe con frequenza $n_i$,  allora la media aritmetica ponderata\index{Media!aritmetica!ponderata}  $M$ dei valori è: \[M=\dfrac{\sum_{i=1}^{n}m_{i}\cdot n_{i}}{\sum_{i=1}^{n} n_{i}}=\dfrac{\sum_{i=1}^{n}m_{i}\cdot n_{i}}{n}\]
\end{defn}
Possiamo quindi costruire la~\vref{tab:Mediaperclassi} e ottenere la media\[M=\dfrac{745}{27}\simeq\num{27.59}\]

\begin{defn}[Varianza ponderata per classi]
		Se abbiamo una distribuzione di $n$ classi di frequenze  e se $m_i$ è il valore centrale di ciascuna classe con frequenza $n_i$
	la varianza  è la media della somma dei quadrato degli scarti dalla media\[\sigma^{2}=\dfrac{\sum_{i=1}^{n}(m_{i}-M)^{2}\cdot n_{i}}{\sum_{i=1}^{n} n_{i}}\] 
\end{defn}
Otteniamo la~\vref{tab:Varianzaperclassi} per cui \[\sigma^2=\dfrac{1518.5187}{27}\simeq\num{56.24}\]
mentre lo scarto quadratico 
\[\sigma=\sqrt{56.24}\simeq\num{7.50}\]